\documentclass{article}%You can define the type of paper here.
%%Useful packages that I commonly use.
\usepackage[numbers]{natbib}%Bibliography package (help at http://merkel.zoneo.net/Latex/natbib.php).
\usepackage{url}%Package to highlight url.
\usepackage{times}%Sets font to be times.
\usepackage{alltt}%Allows the use of verbatim (good for printing out code).
\usepackage{graphicx}%Used to import images.
\usepackage{amsmath, amssymb, amscd}%Contains the AMS expanded math symbols library.
%%For those who want smaller margins, you can use this:
\usepackage[top=1in, bottom=1in, left=1in, right=1in]{geometry}

\begin{document}

%%Title
\title{Simulation and Modeling\\Project I: Molecular Friction}
\author{Name}
\maketitle

%%Makes the paper use two collums
\twocolumn

%%Introduction-------------------------------------------------------------------------------------
\section{Introduction}
Here is a sample of aligned equations with numbers followed by aligned equations without numbers.
\begin{align}
	m^{\prime}_i	&	= m_i + \delta m\\
	m^{\prime}_j	&	= m_j - \delta m\\
	\delta m 			&	= (1 - \lambda)[\epsilon m_j - (1 - \epsilon)m_i]
\end{align}

\begin{align*}
	m^{\prime}_i	&	= m_i + \delta m\\
	m^{\prime}_j	&	= m_j - \delta m\\
	\delta m 			&	= (1 - \lambda)[\epsilon m_j - (1 - \epsilon)m_i]
\end{align*}

%%Method------------------------------------------------------------------------------------------
\section{Method}
You can place in line math notation to show $\sum^{n-1}_{i=0}{i} = \frac{(n-1)n}{2}$. You can also center math notation on its own line.
$$\sum^{n-1}_{i=0}{i} = \frac{(n-1)n}{2}$$

%%Verification------------------------------------------------------------------------------------
\section{Verification of Program}
Sometimes a line can be too long and you'll get a bad box. Try to avoid bad boxes and for a break if needed.

\noindent\url{http://websrv.cs.umt.edu/classes/cs4}

\noindent\url{77/index.php/Distribution_of_Wealth}

%%Data--------------------------------------------------------------------------------------------
\section{Data}
You may have to use a table for your write up.
\begin{table}[h]%%[h] denotes that the table should be placed here if possible.
	\centering
		\begin{tabular}{| l | l | l |}
			\hline
			$\lambda$	&	Error									&	Top $20\%$\\
			\hline
			$0.00$		&	$1.12\normalfont{e-}15$	&	$52.7\%$\\
			$0.25$		&	$2.61\normalfont{e-}15$	&	$42.8\%$\\
			$0.50$		&	$5.59\normalfont{e-}16$	&	$35.5\%$\\
			$0.75$		&	$9.31\normalfont{e-}16$	&	$29.7\%$\\
			$0.90$		&	$1.49\normalfont{e-}15$	&	$25.5\%$\\
			\hline			
		\end{tabular}
\end{table}

You will also probably need to import a graphic.
%%To compile the document, you may want to omit this figure if you don't have the file in its file architecture or change it to a image that you know exists.
\begin{figure}[h]%I appoligize about this figure causing a warning. 
	\centering
		\includegraphics[width=0.48\textwidth]{assignment01/images/fig01.png}%%File location is here. You can do relative or absolute paths. Comment out for Tex to compile correctly (or make up your own fig01.png).
	\label{fig:fig01} % Comment out for Tex to compile correctly (or make up your own fig01.png).
	\caption{$\lambda = 0.00$}%%Caption will place some caption text beneath the figure.
\end{figure}
It should be noted that images and tables are the hardest things to mess around with. You may have to change their width or the position in the paper in which you place them.

%%Analysis---------------------------------------------------------------------------
\section{Analysis}

%%Interpretation---------------------------------------------------------------------
\section{Interpretation}

%%Critique---------------------------------------------------------------------------
\section{Critique}
If you site something in your text, \LaTeX has its own packages to handle bibliographies and citations.

%%Bibliography-----------------------------------------------------------------------
\begin{thebibliography}{1}
	\bibitem{1}Gould, H., Tobochnik, J., \& Christian, W. (2007). \textit{An Introduction to Computer Simulation Methods: Applications to Physical Systems}. San Francisco, CA: Pearson Education, Inc.
\end{thebibliography} 

\end{document}